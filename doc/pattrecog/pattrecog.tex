
\documentclass{beamer}

\usepackage{cmap}                                       % поиск в PDF
\usepackage[T2A]{fontenc}                       % кодировка
\usepackage[utf8]{inputenc}                     % кодировка исходного текста
\usepackage[english,russian]{babel}     % локализация и переносы

\usepackage{graphicx}

%% \usetheme{Berkeley}
%% \usecolortheme{beaver}

\logo{\includegraphics[height=1cm]{bsu_logo_ru.eps}}
\title{Элементы основ распознавания образов}
\author{Кузьмин А.А.}
\institute[]{\url{http://rfe.bsu.by/}}

\begin{document}

\begin{frame}
  \maketitle
\end{frame}

\section{Содержание}

\begin{frame} \label{cont}
  \frametitle{\insertsection}
  
  \begin{enumerate}
  \item Баланс между интуитивным пониманием и математическими формулировками \pause
  \item Требования для понимания курса \pause
    Теория вероятности;
    Линейная алгебра;
    Математический анализ.
  \end{enumerate}
  
  \begin{equation}
    x = [x_1, x_2, x_3]    
  \end{equation}
\end{frame}

\section{Случайный вектор как модель сигнала}

\begin{frame}
  \frametitle{\insertsection}
  \framesubtitle{\insertsubsection}
  \includegraphics[height=5cm]{test.eps}
\end{frame}

\begin{frame}
  \frametitle{\insertsection}
  \framesubtitle{\insertsubsection}
  Закономерность в случайных событиях. Применяется теория вероятности
\end{frame}

\section{Случайные векторы}
\subsection{Нотация}

\begin{frame}
  \frametitle{\insertsection}
  \framesubtitle{\insertsubsection}
  
\end{frame}

\subsection{Моменты}

\begin{frame}
  \frametitle{\insertsection}
  \framesubtitle{\insertsubsection}
  Закономерность в случайных событиях
\end{frame}

\subsection{Функция распределения}

\begin{frame}
  \frametitle{\insertsection}
  \framesubtitle{\insertsubsection}
  Закономерность в случайных событиях
\end{frame}

\section{Многомерное гауссово (нормальное) распределение}

\begin{frame}
  \frametitle{\insertsection}
  \framesubtitle{\insertsubsection}
  Закономерность в случайных событиях
\end{frame}

\subsection{Свойства}

\begin{frame}
  \frametitle{\insertsection}
  \framesubtitle{\insertsubsection}
  Закономерность в случайных событиях
\end{frame}

\subsection{Графическое представление}

\begin{frame}
  \frametitle{\insertsection}
  \framesubtitle{\insertsubsection}
  Закономерность в случайных событиях
\end{frame}

\section{Задача распознавания образов (классификации)}

\section{Байесовский классификатор}
\subsection{Проверка гипотез для двух классов}

\begin{frame}
  \frametitle{\insertsection}
  \framesubtitle{\insertsubsection}
  Закономерность в случайных событиях
\end{frame}

\subsection{Ограничение сверху на Байесовскую ошибку}

\begin{frame}
  \frametitle{\insertsection}
  \framesubtitle{\insertsubsection}
  Закономерность в случайных событиях
\end{frame}

\subsection{Байесовский линейный классификатор}

\begin{frame}
  \frametitle{\insertsection}
  \framesubtitle{\insertsubsection}
  Закономерность в случайных событиях
\end{frame}


\end{document}
